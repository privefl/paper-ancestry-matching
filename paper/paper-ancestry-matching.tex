%% LyX 1.3 created this file.  For more info, see http://www.lyx.org/.
%% Do not edit unless you really know what you are doing.
\documentclass[english, 12pt]{article}
\usepackage{times}
%\usepackage{algorithm2e}
\usepackage{url}
\usepackage{bbm}
\usepackage[T1]{fontenc}
\usepackage[latin1]{inputenc}
\usepackage{geometry}
\geometry{verbose,letterpaper,tmargin=2cm,bmargin=2cm,lmargin=1.5cm,rmargin=1.5cm}
\usepackage{rotating}
\usepackage{color}
\usepackage{graphicx}
\usepackage{subcaption}
\usepackage{amsmath, amsthm, amssymb}
\usepackage{setspace}
\usepackage{lineno}
\usepackage{hyperref}
\usepackage{bbm}
\usepackage{makecell}

\renewcommand{\arraystretch}{1.2}

%\usepackage{xr}
%\externaldocument{SCT-supp}

%\linenumbers
%\doublespacing
\onehalfspacing
%\usepackage[authoryear]{natbib}
\usepackage{natbib} \bibpunct{(}{)}{;}{author-year}{}{,}

%Pour les rajouts
\usepackage{color}
\definecolor{trustcolor}{rgb}{0,0,1}

\usepackage{dsfont}
\usepackage[warn]{textcomp}
\usepackage{adjustbox}
\usepackage{multirow}
\usepackage{graphicx}
\graphicspath{{../figures/}}
\DeclareMathOperator*{\argmin}{\arg\!\min}

\let\tabbeg\tabular
\let\tabend\endtabular
\renewenvironment{tabular}{\begin{adjustbox}{max width=0.9\textwidth}\tabbeg}{\tabend\end{adjustbox}}

\makeatletter

%%%%%%%%%%%%%%%%%%%%%%%%%%%%%% LyX specific LaTeX commands.
%% Bold symbol macro for standard LaTeX users
%\newcommand{\boldsymbol}[1]{\mbox{\boldmath $#1$}}

%% Because html converters don't know tabularnewline
\providecommand{\tabularnewline}{\\}

\usepackage{babel}
\makeatother


\begin{document}


\title{Ancestry inference and grouping\\from principal component analysis of genetic data
}
\author{Florian Priv\'e$^{\text{1,}*}$}

\date{~ }
\maketitle

\noindent$^{\text{\sf 1}}$National Centre for Register-Based Research, Aarhus University, Aarhus, 8210, Denmark. \\
\noindent$^\ast$To whom correspondence should be addressed.\\

\noindent Contact:
\begin{itemize}
\item \url{florian.prive.21@gmail.com}
\end{itemize}

\newpage

\abstract{
}


%%%%%%%%%%%%%%%%%%%%%%%%%%%%%%%%%%%%%%%%%%%%%%%%%%%%%%%%%%%%%%%%%%%%%%%%%%%%%%%%

\newpage

$^\dag$ Further defined in supplementary section ``Defintions''.

\section*{Introduction}

Why do we need to group individuals in ancestry groups?

There is no consensus on the method to use to infer population structure, but particularly on grouping individuals. 

Why it makes sense to use PCA-based distances -> correction for population structure, and geographical distance.

%%%%%%%%%%%%%%%%%%%%%%%%%%%%%%%%%%%%%%%%%%%%%%%%%%%%%%%%%%%%%%%%%%%%%%%%%%%%%%%%

\section*{Measures of genetic dissimilarity between populations}

We first compare four measures of genetic dissimilarity using populations of the 1000 genomes as example \cite[]{10002015global}. 
The $F_{ST}$$^\dag$ is an ubiquitous measure of genetic dissimilarity between populations and the first measure we use in this comparison.
The other three measures compared are distances applied to the PC scores$^\dag$ of the genetic data: 1) the Bhattacharyya distance$^\dag$; 2) the distance between the centers (geometric median$^\dag$) of the two populations; 3) the shorter distance between pairs of PC scores from the two populations.
The distance between population centers seems to be an appropriate PCA-based distance as this (squared) distance is approximately proportional to the $F_{ST}$ (Figure [TODO]) and provides an appropriate clustering of populations (Figure [TODO]).
In contrast, the two other Bhattacharyya and shortest distances do not provide as satisfactory results (Figures [TODO]).

%%%%%%%%%%%%%%%%%%%%%%%%%%%%%%%%%%%%%%%%%%%%%%%%%%%%%%%%%%%%%%%%%%%%%%%%%%%%%%%%

%\section{Material and Methods}



\renewcommand{\thefigure}{S\arabic{figure}}
\setcounter{figure}{0}
\renewcommand{\thetable}{S\arabic{table}}
\setcounter{table}{0}
\renewcommand{\theequation}{S\arabic{equation}}
\setcounter{equation}{0}
\renewcommand{\thesection}{S\arabic{section}}
\setcounter{section}{1}

\section*{Definitions $\dag$}

\begin{itemize}
	
	\item The $\boldsymbol{F_{ST}}$ measures the relative amount of genetic variance between populations compared to the total genetic variance within these populations \cite[]{wright1965interpretation}.
	We use the weighted average formula proposed in \cite{weir1984estimating}, which we now implement in our package bigsnpr \cite[]{prive2017efficient}.
	
	\item The {\bf Principal Component (PC) scores} are defined as $U \Delta$, where $U \Delta V^T$ is the singular value decomposition of the (scaled) genotype matrix. They are usually truncated, e.g.\ corresponding to the first 20 principal dimensions only. 
	
	\item The {\bf Bhattacharyya distance} between two multivariate normal distributions $\mathcal{N}(\boldsymbol\mu_1,\,\boldsymbol\Sigma_1)$ and $\mathcal{N}(\boldsymbol\mu_2,\,\boldsymbol\Sigma_2)$, which is defined as
	$D_B={1\over 8}(\boldsymbol\mu_2-\boldsymbol\mu_1)^T \boldsymbol\Sigma^{-1}(\boldsymbol\mu_2-\boldsymbol\mu_1)+{1\over 2}\log \,\left({|\boldsymbol\Sigma| \over \sqrt{|\boldsymbol\Sigma_1| \, |\boldsymbol\Sigma_2|} }\right)$,
	where $\boldsymbol\Sigma={\boldsymbol\Sigma_1+\boldsymbol\Sigma_2 \over 2}$ and $|M|$ is the absolute value of the determinant of matrix $M$ \cite[]{bhattacharyya1943measure,fukunaga1990introduction}. 
	The mean and covariance parameters for each population are computed using the robust location and covariance parameters as proposed in \cite{prive2020efficient}.
	
	\item The {\bf geometric median} of points is the point that minimizes the sum of all Euclidean distances to these points. We now implement this as function \texttt{geometric\_median} in our R package bigutilsr.
	
\end{itemize}



\subsection{Ancestry matching \label{ancestry}}

We use the 1000 Genomes (1000G) Project data \cite[]{10002015global} to infer the ancestry of individuals from a study dataset using the following steps: 1) we project individuals from the study dataset onto the PCA space computed using the 1000G data \cite[]{prive2020efficient}; 2) we compute the robust center and covariance parameters of the Mahalanobis distance \cite[]{prive2020efficient} for each of the 26 populations of the 1000G data; 3) we compute the Mahalanobis distance for each projected individual of the study data and each population of the 1000G data (using the center and covariance parameters computed previously); 4) we derive p-values from those distances, as they are approximately chi-squared distributed with the number of PCs used for computing the distances as degrees of freedom; 5) we assign each individual the population group with the largest corresponding p-value (smallest distance), but discard individuals whose maximum p-value is less than 0.05 (i.e.\ individuals far from every population of the 1000G data). 


%%%%%%%%%%%%%%%%%%%%%%%%%%%%%%%%%%%%%%%%%%%%%%%%%%%%%%%%%%%%%%%%%%%%%%%%%%%%%%%%

\section{Results}


\subsection{Ancestry matching}

We performed ancestry estimation of the individuals in the UK Biobank using the 1000G data (Section \ref{ancestry}), discarding individuals with unknown ancestry or mixed ancestry.
We did not infer ancestry for 15.9\% of UK Biobank that had a maximum p-value smaller than 0.05 (Figure \ref{fig:ancestry-pval}). 
More precisely, among ``White'', ``British'' and ``Irish'' ancestries, this represented respectively 20.2\%, 11.1\% and 20.3\%, while this represented between 44.5\% (``Chinese'') and 74.3\% (``Bangladeshi'') for other populations. 
In other words, 3 out of 4 of the UK Biobank individual who self-reports as ``Bangladeshi'' could not be attributed to any 1000G population, although 1000G data includes several ``Bengali in Bangladesh'' (SAS\_BEB).
Only 10 people out of 401,048 were wrongly classified in ``super'' population of the 1000G; e.g.\ one Chinese person was classified as European by our method (Table \ref{tab:ancestry-pred}). 
Moreover, our method is able to accurately differentiate between sub-continental populations such as differentiating between Pakistani, Bangladeshi and Chinese people (Table \ref{tab:ancestry-pred}).

% latex table generated in R 3.6.0 by xtable 1.8-4 package
% Mon Oct 21 15:14:34 2019
\begin{table}[ht]
\centering
\caption{Self-reported ancestry (top) of UKBB individuals and their matching to 1000G populations (left) by our method. See the description of 1000G populations at \url{https://www.internationalgenome.org/category/population/}.} 
\label{tab:ancestry-pred}
\begin{tabular}{|l|c|c|c|c|c|c|c|c|c|c|c|c|c|}
  \hline
 & British & Irish & White & Other White & Indian & Pakistani & Bangladeshi & Chinese & Other Asian & Caribbean & African & Other Black & NA \\ 
  \hline
AFR\_ACB &  &  &  &  &  &  &  &  &  & 1832 & 580 & 31 & 259 \\ 
  AFR\_ASW &  &  &  &  &  &  &  &  &  & 315 & 13 & 10 & 39 \\ 
  AFR\_ESN &  &  &  &  &  &  &  &  &  &  & 155 & 1 & 29 \\ 
  AFR\_GWD &  &  &  &  &  &  &  &  &  &  & 20 &  & 4 \\ 
  AFR\_LWK &  &  &  &  &  &  &  &  &  &  & 34 &  & 4 \\ 
  AFR\_MSL &  &  &  &  &  &  &  &  &  &  & 26 &  &  \\ 
  AFR\_YRI &  &  &  &  &  &  &  &  &  & 5 & 469 & 3 & 81 \\ 
   \hline
AMR\_CLM & 1 &  &  & 50 &  &  &  &  &  &  &  &  & 142 \\ 
  AMR\_MXL &  &  & 1 & 3 &  &  &  &  &  &  &  &  & 20 \\ 
  AMR\_PEL &  &  &  & 2 &  &  &  &  &  &  &  &  & 19 \\ 
   \hline
EAS\_CDX &  &  &  &  &  &  &  & 2 &  &  &  &  & 1 \\ 
  EAS\_CHB & 1 &  &  &  &  &  &  & 445 & 10 &  &  &  & 7 \\ 
  EAS\_CHS &  &  &  &  &  &  &  & 380 & 9 &  &  &  & 21 \\ 
  EAS\_JPT &  &  &  &  &  &  &  & 3 & 25 &  &  &  & 122 \\ 
  EAS\_KHV &  &  &  &  &  &  &  & 4 & 16 &  &  &  & 22 \\ 
   \hline
EUR\_CEU & 325725 & 6675 & 350 & 4855 &  &  &  & 1 &  &  &  &  & 1363 \\ 
  EUR\_FIN & 2 &  &  & 118 &  &  &  &  &  &  &  &  & 1 \\ 
  EUR\_GBR & 57497 & 3486 & 79 & 454 &  &  &  &  &  &  & 1 &  & 244 \\ 
  EUR\_IBS & 28 &  & 4 & 452 &  &  &  &  &  &  &  &  & 8 \\ 
  EUR\_TSI & 25 &  & 1 & 177 &  &  &  &  &  &  &  &  & 5 \\ 
   \hline
SAS\_BEB &  &  &  &  & 11 &  & 57 &  &  &  &  &  & 7 \\ 
  SAS\_GIH &  &  &  &  & 334 &  &  &  & 1 &  &  &  & 4 \\ 
  SAS\_ITU & 1 &  &  &  & 314 & 1 &  &  & 264 & 3 &  &  & 82 \\ 
  SAS\_PJL & 1 &  &  &  & 1749 & 767 &  &  & 104 & 2 &  &  & 177 \\ 
  SAS\_STU &  &  &  &  & 38 &  &  &  & 185 &  &  &  & 40 \\ 
   \hline
NA & 47748 & 2594 & 110 & 9705 & 3270 & 980 & 164 & 669 & 1133 & 2140 & 1906 & 73 & 6970 \\ 
   \hline
\end{tabular}
\end{table}

\noindent[OKAY TO CLASSIFY SOME OTHER WHITE AS CENTRAL/SOUTH AMERICANS? MAYBE, SEE NEXT SUBSECTION]

We also  applied our ancestry detection technique to projected individuals from other datasets.
When applied to the European individuals of the POPRES data \cite[]{nelson2008population}, we show that 56.5\% of individuals could not be matched to one of the 26 populations of the 1000G data (Table \ref{tab:ancestry-pred-popres}). 
This is particularly dramatic for all East-Europeans, whose only 11 people could be matched out of 179 (6.1\%).
Moreover, even though 1000G data include several ``Toscani in Italia'' (EUR\_TSI), we were able to match 14.2\% of Italians only.
Nevertheless, all individuals that we could match were identified as of European ancestry. We could also identify accurately sub-regions of Europe; e.g.\ all Spanish and Portugese individuals that we could match (73.1\%) were identified as ``Iberian Population in Spain'' (EUR\_IBS, Table \ref{tab:ancestry-pred-popres}).
Finally, we also projected and matched people from a case-control cohort for celiac disease composed of four European countries: Italy, the Netherlands, the UK and Finland \cite[]{dubois2010multiple}. 
Among UK individuals, 5397 (79.9\%) were matched to either ``Utah Residents with Northern and Western European Ancestry'' (EUR\_CEU) or ``British in England and Scotland'' (EUR\_GBR), 10 (0.2\%) to other European populations, 1 to ``Mexican Ancestry from Los Angeles USA'' (AMR\_MXL) and 1346 (19.9\%) could not be matched (Table \ref{tab:ancestry-pred-celiac}).
Among Finns, 1543 (62.4\%) were matched to ``Finnish in Finland'' (EUR\_FIN), 6 (0.3\%) to other European populations, and 922 (37.3\%) could not be matched.
Among Italians, 229 (22\%) were matched to EUR\_TSI, 15 (1.5\%) to EUR\_IBS and 795 (76.5\%) could not be matched (Table \ref{tab:ancestry-pred-celiac}).

\noindent[USE FIG S1 TO JUSTIFY 5\% THRESHOLD?]

\subsection{Distance between 1000G populations}

\begin{figure}[!htpb]
	\centerline{\includegraphics[width=0.9\textwidth]{hclust-pop-1000G.png}}
	\caption{Dendrogram of PC-based distances between populations of the 1000G data. See the description of 1000G populations at \url{https://www.internationalgenome.org/category/population/}. \label{fig:ancestry-dist}}
\end{figure}

\noindent[WHAT TO SAY ABOUT THIS? DISTANCE INFORMATIVE? SAS CLOSER TO EUR THAN EAS?]


\noindent[DO THIS IN OTHER DATASETS? COULD SHOW THAT EAST EUROPEAN FAR FROM 1000G POPS IN POPRES. WHAT ABOUT UKBB?]

%%%%%%%%%%%%%%%%%%%%%%%%%%%%%%%%%%%%%%%%%%%%%%%%%%%%%%%%%%%%%%%%%%%%%%%%%%%%%%%%

\section{Discussion}

\noindent[DISCUSS LACK OF SOME POPULATIONS IN 1000G]

\noindent[DISCUSS PC-BASED GENETIC DISTANCES BETWEEN POPULATIONS]

\noindent[IMPLICATIONS FOR PRS?]




%%%%%%%%%%%%%%%%%%%%%%%%%%%%%%%%%%%%%%%%%%%%%%%%%%%%%%%%%%%%%%%%%%%%%%%%%%%%%%%%


\clearpage

\section*{Software and code availability}

%R package pcadapt is available on CRAN.
%A tutorial on using pcadapt to detect local adaptation is available at \url{https://bcm-uga.github.io/pcadapt/articles/pcadapt.html}.
%The code used in this paper is available at \url{https://github.com/bcm-uga/pcadapt/tree/master/new-paper/code}.

\section*{Acknowledgements}

\section*{Funding}

F.P.\ is supported by the Danish National Research Foundation (Niels Bohr Professorship to John McGrath).

\section*{Declaration of Interests}

The authors declare no competing interests.

%%%%%%%%%%%%%%%%%%%%%%%%%%%%%%%%%%%%%%%%%%%%%%%%%%%%%%%%%%%%%%%%%%%%%%%%%%%%%%%%

%\newpage

\bibliographystyle{natbib}
\bibliography{refs}

%%%%%%%%%%%%%%%%%%%%%%%%%%%%%%%%%%%%%%%%%%%%%%%%%%%%%%%%%%%%%%%%%%%%%%%%%%%%%%%%

\newpage
\section*{Supplementary Materials}

\renewcommand{\thefigure}{S\arabic{figure}}
\setcounter{figure}{0}
\renewcommand{\thetable}{S\arabic{table}}
\setcounter{table}{0}

%%%%%%%%%%%%%%%%%%%%%%%%%%%%%%%%%%%%%%%%%%%%%%%%%%%%%%%%%%%%%%%%%%%%%%%%%%%%%%%%

%\clearpage

%\subsection*{Supplementary figures}


%%%%%%%%%%%%%%%%%%%%%%%%%%%%%%%%%%%%%%%%%%%%%%%%%%%%%%%%%%%%%%%%%%%%%%%%%%%%%%%%

%\vspace*{1em}

\begin{figure}[!htpb]
\centerline{\includegraphics[width=0.8\textwidth]{hist-pval-max.pdf}}
\caption{Maximum p-values based on robust Mahalabonis distances of UK Biobank individuals from each of the 26 1000G populations.
\label{fig:ancestry-pval}}
\end{figure}

%%%%%%%%%%%%%%%%%%%%%%%%%%%%%%%%%%%%%%%%%%%%%%%%%%%%%%%%%%%%%%%%%%%%%%%%%%%%%%%%

% latex table generated in R 3.6.0 by xtable 1.8-4 package
% Sat Oct 19 00:18:24 2019
\begin{table}[ht]
\centering
\caption{Ancestry (left) of POPRES individuals and their matching to 1000G populations (top) by our method. See the description of 1000G populations at \url{https://www.internationalgenome.org/category/population/}.} 
\label{tab:ancestry-pred-popres}
\begin{tabular}{|l|c|c|c|c|c|}
  \hline
 & EUR\_CEU & EUR\_GBR & EUR\_IBS & EUR\_TSI & NA \\ 
  \hline
Anglo-Irish Isles & 159 & 57 &  &  & 50 \\ 
  Belgium & 28 & 3 &  &  & 12 \\ 
  Central Europe & 3 &  &  & 1 & 51 \\ 
  Eastern Europe & 2 &  &  &  & 28 \\ 
  France & 14 &  & 13 &  & 62 \\ 
  Germany & 37 &  &  &  & 34 \\ 
  Italy &  &  & 12 & 19 & 188 \\ 
  Netherlands & 10 & 2 &  &  & 5 \\ 
  Scandinavia & 5 &  &  &  & 10 \\ 
  SE Europe &  &  &  & 5 & 89 \\ 
  SW Europe &  &  & 193 &  & 71 \\ 
  Switzerland & 33 & 1 & 6 &  & 182 \\ 
   \hline
\end{tabular}
\end{table}

% latex table generated in R 3.6.0 by xtable 1.8-4 package
% Sat Oct 19 00:28:24 2019
\begin{table}[ht]
\centering
\caption{Ancestry (left) of Celiac individuals and their matching to 1000G populations (top) by our method. See the description of 1000G populations at \url{https://www.internationalgenome.org/category/population/}.} 
\label{tab:ancestry-pred-celiac}
\begin{tabular}{|l|c|c|c|c|c|c|c|}
  \hline
 & AMR\_MXL & EUR\_CEU & EUR\_FIN & EUR\_GBR & EUR\_IBS & EUR\_TSI & NA \\ 
  \hline
Finland &  & 4 & 1543 & 1 &  & 1 & 922 \\ 
  Italy &  &  &  &  & 15 & 229 & 795 \\ 
  Netherlands &  & 802 &  & 382 & 1 &  & 463 \\ 
  UK & 1 & 2865 & 1 & 2532 & 4 & 5 & 1346 \\ 
   \hline
\end{tabular}
\end{table}


\end{document}
